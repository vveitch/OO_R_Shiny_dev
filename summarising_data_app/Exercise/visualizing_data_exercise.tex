%% LyX 2.0.6 created this file.  For more info, see http://www.lyx.org/.
%% Do not edit unless you really know what you are doing.
\documentclass[oneside,english]{amsart}
\usepackage[T1]{fontenc}
\usepackage[latin9]{inputenc}
\usepackage{amsthm}

\makeatletter
%%%%%%%%%%%%%%%%%%%%%%%%%%%%%% Textclass specific LaTeX commands.
\numberwithin{equation}{section}
\numberwithin{figure}{section}

\makeatother

\usepackage{babel}
\begin{document}

\title{Module 1 Exercise}

\maketitle
Recall that the skeleton data was collected in order to assess the
accuracy of methods used by anthropologists to estimate age at death.
The Visualizing Skeleton Data RShiny app lets us quickly inspect the
data by producing a number of helpful visualizations. In this exercise
we will do the first stages of an exploratory analysis of this data.
\begin{enumerate}
\item Real world data often comes with mistakes or mislabellings. The skeleton
data includes BMI as both a quantitative variable and as a categorical
variable. Try several different ways of plotting the quantitative
BMIquant blocked by the subgroup BMIcat. Do you notice anything unusual?
\item If our data was overwhelmingly from males or overwhelmingly from obese
people we might miss seeing effects due to these factors. Try plotting
the categorical variables in two different ways. Does it seem like
any group is strongly overrepresented or underrepresented? 
\item Pie charts are popular amongst elementary school students but not
amongst statisticians. Try producing pie charts of Sex and BMIcat
and eyeballing the relative frequencies of each category visually.
The true relative frequencies are shown in the Data Summary under
the chart. Were your estimates accurate?
\item Data that comes with several covariates might lead us to conflate
the source of different effects. For example, if almost all of the
obese people in the dataset were very old then we might misattribute
to age inaccurate predictions due to obesity. Use the plots to explore
whether this phenomena is present in our data set. Are there comparisons
you would like to make that are not possible with the plotting done
in the app?
\item What is the average BMI for females in this data set? What is the
average BMI for males?
\item For each of the estimation strategies (Di Gangi et al. and Suchey-Brooks)
decide whether the method tends to overestimate or underestimate age
of death. The errors in estimation are given as DGerror and SBerror
respectively. What plots are most useful for assessing this? If you
were an anthropologist how might you make use of this information?
\item For each of the estimation strategies, does body mass or sex substantially
impact the accuracy of the estimate? How about the variability of
the estimate? What kind of plots are useful for assessing this?
\item In practical terms an anthropologist would just like to know which
of these methods they should use. What further exploratory analysis
would you like to do before making a preliminary recommendation? Are
there any additional visualizations or plots that you think should
be produced?\end{enumerate}

\end{document}
